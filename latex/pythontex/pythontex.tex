% Showcasing PythonTeX
% Some examples are taken from
% http://mirrors.ctan.org/macros/latex/contrib/pythontex/pythontex_gallery.pdf

\documentclass{scrartcl}

% pdflatex
\ifx\pdfmatch\undefined
\else
    \usepackage[T1]{fontenc}
    \usepackage[utf8]{inputenc}
\fi
% xetex:
\ifx\XeTeXinterchartoks\undefined
\else
    \usepackage{fontspec}
    \defaultfontfeatures{Ligatures=TeX}
\fi
% luatex:
\ifx\directlua\undefined
\else
    \usepackage{fontspec}
\fi
% End engine-specific settings

\usepackage[parfill]{parskip}
\usepackage{amsmath,amssymb}
\usepackage{fullpage}
\usepackage{graphicx}
\usepackage[svgnames]{xcolor}
\usepackage{url}
\urlstyle{same}

\usepackage[makestderr]{pythontex}
\restartpythontexsession{\thesection}

\newcommand{\pytex}{Python\TeX}
\title{\pytex\ in CoCalc}
\author{Name of Author}

\begin{document}

\maketitle

CoCalc supports running \pytex\ automatically for documents that load the \verb|pythontex| package.
It executes it using the \verb|pythontex3| engine for Python 3.

\section{\pytex\ Examples}

\begin{pycode}
import math
print(r'Pi = {}'.format(math.pi))
\end{pycode}

A huge number is \py{2*3**45 - 1}.

And a sum of unicode characters: \py{"ä"+"µ"+"ß"}.

\subsection{A simple formula}

$\frac{3}{\pi} = \py{3 / math.pi}$.

\subsection{Native SymPy Support}

Via the \verb|sympy|  command and inside a \verb|sympyblock|.

\begin{sympyblock}
var('x, y, z')
z = x + y
\end{sympyblock}

Now we can access what $z$ is equal to:

\[z=\sympy{z}\]

Many things are possible, including some very nice calculus.
First, define the integral expression $g$ and then us \verb|g.doit()| to run it inside a \LaTeX\ formula.

\begin{sympyblock}
f = x**3 + cos(x)**5
g = Integral(f, x)
\end{sympyblock}

\[\sympy{g}=\sympy{g.doit()}\]

It's easy to use arbitrary symbols in equations.

\begin{sympyblock}
phi = Symbol(r'\phi')
h = Integral(exp(-phi**2), (phi, 0, oo))
\end{sympyblock}

\[\sympy{h}=\sympy{h.doit()}\]

\pagebreak

\subsection{Code block with a plot}

The session is \verb|pycode| to be distinct from the \verb|default| session.
This allows \pytex\ to speed up the execution by running the sympy code and the plot below in parallel.

\begin{pycode}[pycode]
from pylab import *
# Define f(t), the desired function to plot
def f(t):
  return cos(2 * pi * exp(t)) * exp(-t)
# Generate the points (t_i, y_i) to plot
t = linspace(0, 4, 1000)
y = f(t)
# Begin with an empty plot, 5 x 3 inches
clf()
figure(figsize=(5, 3))
# Use TeX fonts
rc("text", usetex=True)
# Generate the plot with annotations
plot(t, y)
title("Damped exponential decay")
text(2.5, 0.4, r"$y = \cos(2 \pi \exp(t)) e^{-t}$")
xlabel("time (s)")
ylabel("voltage (mV)")
# Save the plot as a PDF file
savefig("myplot.pdf", bbox_inches="tight")
# Include the plot in the current LaTeX document
print(r"\begin{center}")
print(r"\includegraphics[width=0.85\textwidth]{myplot.pdf}")
print(r"\end{center}")
\end{pycode}

\section{Error Handling}

CoCalc also displays the first processing error next to the line \pytex\ reports.
The example below should have such a ``bug'' indicator.

\pytex\ also allows code to be typeset next to the stderr it produces.
This requires the package option \verb|makestderr|.

\begin{pyblock}[errorsession][numbers=left]
x = 123
y = 345
z = x + y +
\end{pyblock}

This code causes a syntax error:

\stderrpythontex[verbatim][frame=single]

\section{Learn more}

Package documentation: \url{https://ctan.org/pkg/pythontex}

\end{document}
